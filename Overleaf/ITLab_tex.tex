% Параметры компиляции

\documentclass[a4document]{article}

\usepackage[utf8x]{inputenc}
\usepackage[english,russian]{babel}
\usepackage{enumitem}
\usepackage{cmap}

\thispagestyle{empty} 

\usepackage[14pt]{extsizes}
\usepackage[left=2cm,right=2cm,
    top=2cm,bottom=2cm]{geometry}
\usepackage[unicode, pdftex]{hyperref}
\usepackage{graphicx}

\begin{document}

% Титульный лист
{
\centering{
Министерство образования и науки Российской Федерации Федеральное государственное автономное образовательное учреждение высшего профессионального образования «Национальный исследовательский Нижегородский государственный университет им. Н.И. Лобачевского»
\newline
Институт информационных технологий, математики и механики
\bigbreak
\bigbreak
\bigbreak
\bigbreak
\bigbreak
\bigbreak
\bigbreak
\bigbreak
\bigbreak
\textbf{Отчет по программному проекту "IT-перспектива" \\ 
"Устройство для наблюдения за состоянием здоровья человека в рабочее время"}


\bigbreak
\bigbreak
\bigbreak
\bigbreak
}

\begin{flushright}
    \textbf{Выполнили: } \\
    студент группы 382006-1 \\
    Юнин Д.Д. \\
    студент группы 382008-1 \\
    Булгаков Д.Э. \\
    \bigbreak
    \textbf{Ментор: }  \\ 
    Карчков Д.А.
\end{flushright}

\vspace*{\fill}

\begin{center}
Нижний Новгород \\
2022 г.
\end{center}
}

% Оглавление
{
\newpage

\begin{flushleft}
\tableofcontents%
\end{flushleft}
}

% Введение 
{
\newpage
\section*{Введение.} \addcontentsline{toc}{section}{Введение.}

\par
Наблюдение за состоянием здоровья необходимо для обеспечения гарантии первоначальной и последующей физической пригодности работников для выполнения поставленных перед ними профессиональных задач. На данный момент для контроля здоровья сотрудников применяется обязательный периодический медицинский осмотр с периодом проведения один раз в год. Однако, такого вида обследования может быть не достаточно в случаях, если специальность сотрудника связана с : 

\begin{itemize}
    \item вредными и(или) опасными производственными факторами.
    \item использованием технически сложных механизмов и устройств повышенной опасности.
    \item пищевой промышленностью.
    \item изменением условий труда.
\end{itemize} 

\par\noindent 
В таких случаях необходимо проводить регулярный мониторинг здоровья сотрудника, чтобы отслеживать динамику его физического и психологического состояния и свести к минимуму вред, причиненный здоровью и трудовому потенциалу работника.

\par\noindent
На текущий момент, существуют устройства личного пользования для мониторинга ключевых показателей жизненно важных функций. 
(фитнес-браслеты, умный часы и т.п.) На крупных производствах и в больших организациях наиболее распространены 
только устройства контроля рабочего времени сотрудника.

\par\noindent
Необходимо создать устройство, для регулярного определения состояния здоровья человека, 
которое в зависимости от потребности компаний может иметь разный набор датчиков измерения показателей,
а также с понятной расшифровкой полученных данных в мобильном приложении. 

}

% Постановка задачи и цели работы
{
\newpage
\section*{Постановка задачи и цели работы.} \addcontentsline{toc}{section}{Постановка задачи и цели работы.}

\par\noindent
\textbf{Задача :}
\newline
Разработать устройство, для регулярного определения состояния здоровья человека, 
которое в зависимости от потребности компаний может иметь разный набор датчиков измерения показателей,
а также с понятной расшифровкой полученных данных в мобильном приложении. 

\bigbreak
\par\noindent
\textbf{Цели :}
\newline
Необходимо разработать : 
\begin{enumerate}
    \item Основные модули устройства : 
    \begin{itemize}
        \item модуль с экраном.
        \item модуль с микроконтроллером.
        \item модуль с датчиками. 
        \item модуль с батареей.
    \end{itemize}
    \item ПО микроконтроллера для считывания информации с датчиков и передачи на сервер.
    \item ПО для обработки полученной информации и получения диагноза.
\end{enumerate}

}

% Методы решения задачи
{
\newpage
\section*{Методы решения задачи.} \addcontentsline{toc}{section}{Методы решения задачи.}

\subsection*{Модули устройства.}
{
Модули устройства будут изготовлены из пластика с использованием 3D принтера.
Создание 3D моделей в AutoDesk TinkerCad.

Датчики, которые будут использоваться : 
\begin{itemize}
    \item 
\end{itemize}

Микроконтроллер : ESP32

}

\subsection*{ПО микроконтроллера для считывания информации с датчиков и передачи на сервер.}
{
Для написание программы микроконтроллера используется фреймворк Arduino.
}

\subsection*{ПО для обработки полученной информации и получения диагноза.}
{}

}

% Программная реализация (высокоуровневая архитектура, описание основных алгоритмов и структур данных…)
{
\newpage
\section*{Программная реализация.} \addcontentsline{toc}{section}{Программная реализация.}
}

% Результаты работы (описание выполненной процедуры тестирования, численные результаты)
{
\newpage
\section*{Результаты работы.} \addcontentsline{toc}{section}{Результаты работы.}
}

% Руководство пользователя
{
\newpage
\section*{Руководство пользователя.} \addcontentsline{toc}{section}{Руководство пользователя.}
}

% Заключение (основные результаты)
{
\newpage
\section*{Заключение.} \addcontentsline{toc}{section}{Заключение.}
}

% Список литературы
{
\newpage
\section*{Список литературы.} \addcontentsline{toc}{section}{Список литературы.}
}

% Приложения (если есть)
{
\newpage
\section*{Приложения.} \addcontentsline{toc}{section}{Приложения.}
}


\end{document}